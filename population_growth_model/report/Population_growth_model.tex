\documentclass[12pt, a4paper]{article}

%%%%%%%%%%%%%%%紙張大小設定%%%%%%%%%%%%%%%
% \paperwidth=65cm
% \paperheight=160cm

%%%%%%%%%%%%%%%引入Package%%%%%%%%%%%%%%%
\usepackage[margin=2cm]{geometry} % 上下左右距離邊緣2cm
\usepackage{amsmath,amsthm,amssymb} % 引入 AMS 數學環境
%\usepackage{yhmath}      % math symbol
\usepackage{caption3}    % caption 增強
\usepackage{setspace}    % 控制空行
\usepackage{fontspec}    % 加這個就可以設定字體
\usepackage{type1cm}	 % 設定fontsize用
\usepackage{titlesec}   % 設定section等的字體
\usepackage{titling}    % 加強 title 功能
\usepackage{fancyhdr}   % 頁首頁尾
\usepackage[square, comma, numbers, sort&compress]{natbib}
% cite加強版
\usepackage[usenames, dvipsnames]{color}  % 可以使用顏色
\usepackage{hyperref}
% ref加強版
%\usepackage{soul}       % highlight
%\usepackage{ulem}       % 字加裝飾
%\usepackage{framed}     % 可以加文字方框
\usepackage{enumerate}  % 加強版enumerate
\usepackage{stmaryrd}
\usepackage{appendix}
%\usepackage{quotchap}
\usepackage[USenglish]{babel}
\usepackage[nodayofweek,level]{datetime}

%%%%%%%%%%%%%%圖形用%%%%%%%%%%%%%%%
\usepackage{graphicx}    % 圖形插入用
\usepackage{subcaption}
%\graphicspath{{images/}}  % 搜尋圖片目錄
%\usepackage{wrapfig}     % 文繞圖
%\usepackage{floatflt}    % 浮動 figure
%\usepackage{float}       % 浮動環境
%\usepackage{subfig}      % subfigures

%%%%%%%%%%%%%%各式各樣的 table package 和 caption%%%%%%%%%%%%%%%
\usepackage{longtable}
\usepackage{tabu}
\usepackage{multirow}
\usepackage{array}
\usepackage{booktabs}
\usepackage{tabularx}
\usepackage{caption}
\usepackage{ltcaption}

%%%%%%%%%%%%%%pseudocode 用的 package%%%%%%%%%%%%%%%
\usepackage{algorithm}
\usepackage{algorithmicx}
\usepackage[noend]{algpseudocode}

%%%%%%%%%%%%%%程式碼package%%%%%%%%%%%%%%%
\usepackage{listings}
\usepackage{lstautogobble}

%%%%%%%%%%%%%%中文 Environment%%%%%%%%%%%%%%%
\usepackage[AutoFakeBold, CheckSingle, CJKmath]{xeCJK}  % xelatex 中文
\usepackage{CJKulem}	% 中文字裝飾
%\setCJKmainfont{DFKai-SB}
%\setCJKsansfont{DFKai-SB}
%\setCJKmonofont{DFKai-SB}
%設定中文為系統上的字型,而英文不去更動,使用原TeX字型

%\XeTeXlinebreaklocale "zh"             %這兩行一定要加,中文才能自動換行
%\XeTeXlinebreakskip = 0pt plus 1pt     %這兩行一定要加,中文才能自動換行

%%%%%%%%%%%%%font size setting%%%%%%%%%%%%%%%
%\def\footnotesize{\fontsize{8}{9.5}\selectfont}
%\def\small{\fontsize{10}{13}\selectfont}
%\def\normalsize{\fontsize{11}{18}\selectfont}
%%\def\large{\fontsize{14}{20}\selectfont}
%\def\Large{\fontsize{15}{21}\selectfont}
%\def\LARGE{\fontsize{20}{30}\selectfont}
% \def\huge{\fontsize{34}{51}\selectfont}
% \def\Huge{\fontsize{38}{57}\selectfont}

%%%%%%%%%%%%%%%Theme Input%%%%%%%%%%%%%%%%
%\input{themes/chapter/neat}

%%%%%%%%%%%titlesec settings%%%%%%%%%%%%%%
%\titleformat{\part}{\bf}
            %{\arabic{Section}}{1em}{\fontsize{17}{20}\selectfont}
%\titleformat{\section}{\centering\Large}
            %{\arabic{section}}{0em}{}
%\titleformat{\subsection}{\large}
            %{\arabic{subsection}}{0em}{}
%\titleformat{\subsubsection}{\bf\normalsize}
            %{\arabic{subsubsection}}{0em}{}
%\titleformat{command}[shape]{format}{label}
            %{編號與標題距離}{before}[after]

%%%%%%%%%%%%variable settings%%%%%%%%%%%%%%
%\numberwithin{equation}{section}
%\setcounter{secnumdepth}{4}  %章節標號深度
%\setcounter{tocdepth}{1}  %目錄深度

%%%%%%%%%%%%%%%頁面設定%%%%%%%%%%%%%%%
\setlength{\headheight}{15pt}  %with titling
\setlength{\droptitle}{-1.5cm} %title 與上緣的間距
\parindent=24pt %設定縮排的距離
\parskip=0.8em  %設定行距
%\pagestyle{empty}  % empty: 無頁碼
%\pagestyle{fancy}  % fancy: fancyhdr

%use with fancygdr
%\lhead{\leftmark}
%\chead{}
%\rhead{}
%\lfoot{}
%\cfoot{}
%\rfoot{\thepage}
%\renewcommand{\headrulewidth}{0.4pt}
%\renewcommand{\footrulewidth}{0.4pt}

%\fancypagestyle{firststyle}
%{
  %\fancyhf{}
  %\fancyfoot[C]{\footnotesize Page \thepage\ of \pageref{LastPage}}
  %\renewcommand{\headrule}{\rule{\textwidth}{\headrulewidth}}
%}

%%%%%%%%%%%%%%%重定義一些command%%%%%%%%%%%%%%%
\renewcommand{\refname}{}  %設定參考資料的標題名稱
\renewcommand{\abstractname}{\LARGE Abstract} %設定摘要的標題名稱
\renewcommand{\appendixname}{Annexe}
\setlength{\bibsep}{4.8pt}
%\renewcommand{\thesection}{\Roman{section}}


%%%%%%%%%%%%%%%特殊功能函數符號設定%%%%%%%%%%%%%%%
%\newcommand{\citet}[1]{\textsuperscript{\cite{#1}}}
\newcommand{\np}[1]{\\[{#1}] \indent}
\newcommand{\transpose}[1]{{#1}^\mathrm{T}}
\newcommand{\adj}{\mathrm{adj}}
%%%% Geometry Symbol %%%%
\newcommand{\degree}{^\circ}
\newcommand{\Arc}[1]{\wideparen{{#1}}}
\newcommand{\Line}[1]{\overleftrightarrow{{#1}}}
\newcommand{\Ray}[1]{\overrightarrow{{#1}}}
\newcommand{\Segment}[1]{\overline{{#1}}}

%%%%%%%%%%%%%%%證明、結論、定義等等的環境%%%%%%%%%%%%%%%
\renewcommand{\proofname}{\bf 證明:} %修改Proof 標頭
\newtheoremstyle{mystyle}% 自定義Style
  {12pt}{12pt}%     上下間距
  {}%               內文字體
  {}%               縮排
  {}%               標頭字體
  {}%               標頭後標點
  {0.4em}%          內文與標頭距離
  {}%               Theorem head spec (can be left empty, meaning 'normal')

% 改用粗體,預設 remark style 是斜體
\theoremstyle{mystyle}	% 定理環境style
\newtheorem{theorem}{定理}[section]
\newtheorem{definition}{定義}[section]
\newtheorem{formula}{公式}
\newtheorem{condition}{條件}
\newtheorem{supposition}{假設}
\newtheorem{conclusion}{結論}
\newtheorem{lemma}{引理}[section]
\newtheorem{property}{性質}[section]
\newtheorem{corollary}{推論}[section]
\newtheorem{example}{ex}
%\linespread{1.3} %調整行距

%%%%%%%%%%%%%%%虛擬碼關鍵字改法文 調間距%%%%%%%%%%%%%%%
\let\Algorithm\algorithm
\renewcommand\algorithm[1][]{\Algorithm[#1]\setstretch{1.2}}
\renewcommand{\algorithmicif}{\textbf{Si}}
\renewcommand{\algorithmicthen}{\textbf{alors}}
\renewcommand{\algorithmicfor}{\textbf{Pour}}
\renewcommand{\algorithmicdo}{\textbf{faire}}
\renewcommand{\algorithmicprocedure}{\textbf{Proc\'edure}}
\renewcommand{\algorithmicelse}{\textbf{Sinon}}
%\algrenewtext{EndProcedure}{}
%\algrenewtext{EndFor}{}
%\algrenewtext{EndIf}{}

%%%%%%%%%%%%%%%設定 hyperref%%%%%%%%%%%%%%%
\hypersetup{
unicode=true, 
bookmarksdepth=-1, 
pdfborder={0 0 0}, 
colorlinks=true, 
linkcolor=RoyalBlue,
citecolor=blue
}


%%%%%%%%%%%%%%%autoref 改法文%%%%%%%%%%%%%%%
\renewcommand{\tableautorefname}{Tableau}
\def\appendixautorefname{annexe}



%%%%%%%%%%%%%%%algorithm需要兩行caption時%%%%%%%%%%%%%%%
\DeclareCaptionFormat{algorithm}{\vspace{0.1em}{%
  \parbox[c][3em][c]{\textwidth}{\hspace{0.2em}#1#2#3}}}

%%%%%%%%%%%%%%%table整列要粗體%%%%%%%%%%%%%%%
\newcolumntype{+}{>{\global\let\currentrowstyle\relax}}
\newcolumntype{^}{>{\currentrowstyle}}
\newcommand{\rowstyle}[1]{\gdef\currentrowstyle{#1}%
  #1\ignorespaces
}

%%%%%%%%%%%%%%%設定欄寬特殊情況 (不同於 m p...)%%%%%%%%%%%%%%%
\newcolumntype{L}[1]{>{\raggedright\let\newline\\\arraybackslash\hspace{0pt}}m{#1}}
\newcolumntype{C}[1]{>{\centering\let\newline\\\arraybackslash\hspace{0pt}}m{#1}}
\newcolumntype{R}[1]{>{\raggedleft\let\newline\\\arraybackslash\hspace{0pt}}m{#1}}

%%%%%%%%%%%%%%%wide longtable%%%%%%%%%%%%%%%
\setlength{\LTleft}{-20cm plus -1fill}
\setlength{\LTright}{\LTleft}


%%%%%%%%%%%%%%%格內分行 基本間格設定%%%%%%%%%%%%%%%
\newcommand{\tabincell}[2]{\begin{tabular}{@{}#1@{}}#2\end{tabular}}
\renewcommand{\arraystretch}{1.5} 
\tabcolsep=9.8pt 

%%%%%%%%%%%%%%%figure caption%%%%%%%%%%%%%%%
\captionsetup[figure]{
 	font = small,
	name = \textsc{Figure},
	justification=justified,
	labelsep=colon,%%%%%%
	skip = \medskipamount}
\captionsetup[sub]{font = small}
\setlength{\belowcaptionskip}{-3pt}

%%%%%%%%%%%%%%%table caption%%%%%%%%%%%%%%%
\captionsetup[table]{
 	font = small,
	%textfont = sc, 
	name = \textsc{Tableau}, 
	justification=justified,
	singlelinecheck=false,%%%%%%% a single line is centered by default
	labelsep=colon,%%%%%%
	skip = \medskipamount}



%%%%%%%%%%%%%%%同下面%%%%%%%%%%%%%%%
\DeclareFixedFont{\ttb}{T1}{txtt}{bx}{n}{9} % for bold
\DeclareFixedFont{\ttm}{T1}{txtt}{m}{n}{9}  % for normal
% Defining colors
\definecolor{deepblue}{rgb}{0,0,0.5}
\definecolor{deepred}{rgb}{0.6,0,0}
\definecolor{deepgreen}{rgb}{0,0.5,0}


%%%%%%%%%%%%%%%引入 python用%%%%%%%%%%%%%%%
\newcommand\pythonstyle{\lstset{
  language=Python,
  backgroundcolor=\color{white}, %%%%%%%
  basicstyle=\small\ttfamily,
  %otherkeywords={self},       
  keywordstyle=\ttb\color{deepblue},
  emph={self},          
  emphstyle=\ttb\color{deepred},    
  stringstyle=\color{deepgreen},
  commentstyle=\color{red},  %%%%%%%%
  frame=tb,                         
  showstringspaces=false,
  autogobble=true, 
  breaklines = true,
  tabsize = 4,
  linewidth= \linewidth, 
  %numbers = left,
  columns=fullflexible,
  inputencoding=utf8,
  extendedchars=true,
  literate={á}{{\'a}}1 {ã}{{\~a}}1 {é}{{\'e}}1 {è}{{\`e}}1 {ç}{\c{c}}1
}}

% Python environment
\lstnewenvironment{python}[1][]
{\pythonstyle
\lstset{#1}}{}

\newcommand\pythonexternal[2][]{{
\pythonstyle
\lstinputlisting[#1]{#2}}}


%%%%%%%%%%%%%%%圖檔位置%%%%%%%%%%%%%%%
\graphicspath{
{../figures/},
}


%%%%%%%%%%%%%%%Title的資訊%%%%%%%%%%%%%%%
\title{} %標題
\author{} %作者
\date{} %日期

\begin{document}
%\maketitle %製作tilte page
%\tableofcontents %目錄
%%%%%%%%%%%%%%%%%%%include file here%%%%%%%%%%%%%%%%%%%%%%%%%

%mainfile: Population_growth_model.tex

\pagestyle{fancy} 
\lhead{Atelier Nueromodélisation, 2017}
\rhead{Problem set \#1}
\rfoot{\thepage}
\cfoot{}
\lfoot{~\theauthor}
\renewcommand{\headrulewidth}{0.4pt}
\renewcommand{\footrulewidth}{0.4pt}

\title{Problem Set \#1: Population Growth Model \vspace{-0.5em}}
%\preauthor{} \postauthor{} 
\author{Hsieh Yu-Guan}
\selectlanguage{USenglish}
\date{\formatdate{10}{3}{2017}}
%\date{\today}
\maketitle

\thispagestyle{fancy}

\section{Problem description}

Here we want to model the growth of an animal population. Every year, 
the growth of the population is determined according to its original size.
The population value of each year that we get will be stored in an array 
$p$ so that we can plot its evolution over time.

\section{When growth rate $\alpha$ stays constant}

As a first approximation, we suppose that the growth rate $\alpha$ is contant.
That is, for all $n$, we can write $p_n = p_{n-1} + \alpha p_{n-1}$, where 
$p_n$ is the population value of the $n^{th}$ year (counting from zero) and 
$\alpha$ doesn't depend on $n$.

At first, we fix $p_0 = 2$ and $\alpha = 0.1$ and simulate the population 
growth over a hundred years. The result is shown below:

\vspace{-1em}
\begin{figure}[H]
  \centering
  \includegraphics[width=0.7\linewidth]{fig1}
  \caption{Population growth with $p_0 = 2$ and $\alpha = 0.1$}
\end{figure}

As we can see, the population grows exponentially and explodes very fast, 
$10^{32}$ individuals at the end of a century! But it's somehow not very 
surprising because it's exactly what our equation indicates as it can also
be written in the form $p_n = (1+\alpha)p_{n-1}$. Now let's change the value 
of $\alpha$ and see what happens.

\vspace{-1em}
\begin{figure}[H]
  \centering
  \includegraphics[width=0.7\linewidth]{fig2}
  \caption{Population growth with $p_0 = 2$ and different $\alpha$}
\end{figure}

The population grows more or less fast when $\alpha$ varies. I always keep
$\alpha \le 0.1$ in this figure because if $\alpha$ gets even bigger, the 
population grows really fast and we will not be able to see other curves
with smaller $\alpha$ values.

\vspace{-1em}
\begin{figure}[H]
  \centering
  \includegraphics[width=0.7\linewidth]{fig3}
  \caption{Population growth with $\alpha = 0.1$ and different $p_0$}
\end{figure}

In constrast, as shown in the figure above, changing initial population has 
a smaller impact. In fact, the ratio between two different population values
stays constant over time.

\section{Towards a more realistic model}

Nevertheless, the model proposed in the last section is not very satisfying
since the animal population is not meant to grow forever without limitation.
For instance, the problem of resources should be considered, we need to 
modify our model such that $\alpha$ becomes a funcion of $p$, and of course
we want $\alpha$ to decrease when $p$ increases. Let's choose 
$\alpha = 200 - p$.

\vspace{-1em}
\begin{figure}[H]
  \centering
  \includegraphics[width=0.7\linewidth]{fig4}
  \caption{Plot of $\alpha = 200 - p$}
\end{figure}

It's quite nice. $\alpha$ is a decreasing function, which means that the 
population will grow more and more slowly as time goes by. It gets even
negative when $p$ is greater than $200$. This prevents the population
from growing too large. However, the real value of $\alpha$ here is too big, 
hence we would rather use $\delta_n = 0.001p_{n-1}(200 - p_{n-1})$ 
(in this case $\alpha = 0.001(200-p_{n-1})$). We can now plot the 
varaiation of the population.

\vspace{-1em}
\begin{figure}[H]
  \centering
  \includegraphics[width=0.7\linewidth]{fig5}
  \caption{Population growth with the law 
           $p_n = p_{n-1} + 0.001p_{n-1}(200 - p_{n-1})$}
\end{figure}

It's closer to the reality this time. We obtain a curve with a sigmoid 
shape. The population grows quite fast at the beginning, but then 
the growth slows down and becomes almost linear, and finally after about
fifty years, the population gets saturated with $p \sim 200$. We can still
try to change different parameters in the equation, like the coefficient
before $200-p$ in $\alpha$.

\vspace{-1em}
\begin{figure}[H]
  \centering
  \includegraphics[width=0.7\linewidth]{fig6}
  \caption{Population growth with some different parameters in the law}
\end{figure}

In most of the cases, we see something that is already obeserved: bigger the 
$\alpha$, faster the population grows. Nonetheless, things get more 
interesting when this coefficient becomes bigger ($0.01$ here). An oscillation
behavior appears since the population size can exceed the environment limit
from time to time (but if we keep increasing this parameter, the curve will 
not make sense anymore). At the end of the report, we'd like to see the 
influence of $p_0$ in this model.

\vspace{-1em}
\begin{figure}[H]
  \centering
  \includegraphics[width=0.7\linewidth]{fig7}
  \caption{Population growth with the same law but different $p_0$}
\end{figure}

It turns out that the result is not that different from what we get from
the first model, but when $p_0$ is larger, the population growth could
become slower from the very beginning (say, we're already in the linear
phase). Furthermore, if $p_0$ is greater than $200$, there aren't 
enough environmental resources for all the individuals, and the
population value will decrease to fit this limitation. A case particular
is when $p_0$ is exactly $200$, the population size will be fixed
from the beginning at $200$.

\section{Conclusion}

In conclusion, we have tried to simulate the animal population growth with
two different models. In the first one, we consider the population growth
rate as a constant and see that the population size can explode very
quickly. Therefore, we decide to take account of resource issues in the
second model. With this extra condition, population saturation is reached
after a period of time and too large population gets penalized.

%%%%%%%%%%%%%%%%%%%%%%%%%%%%%%%%%%%%%%%%%%%%%%%%%%%%%%%%%%%%%
%\begin{thebibliography}{99}
%\bibitem[1]{ex}\verb|http://www.example.com/|
%\end{thebibliography}
\end{document}
